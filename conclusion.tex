%!TEX root = paper.tex
\section{Conclusion and Future Work}
This paper presents a new algorithm that generates the match pairs for message passing programs in the context of a CTP.
First, the algorithm sections each process where each section contains roughly $k$ sends from each sender that may match the same number of receives in the section. The bound $k$ is a user input.
The algorithm then approximates the match pairs for each section \cite{DBLP:conf/kbse/HuangMM13}. The key insight of the algorithm in this paper is that the match pairs for each section are generated independently. 
%A send and a receive from two different sections can not be considered for matching. 
This paper presents that all the precise match pairs for a CTP can be generated with the \textit{max bound} of $k$. Experiments demonstrate that all the properties in the benchmarks can be efficiently witnessed with a low $k$-bound. Experiments also show that the algorithm scales to a program that employs a high degree of message non-determinism and/or a high degree of deep communication.

The algorithm in this paper is restricted to the CTP abstraction that only reveals the behavior from an execution trace given a concrete input. The program structures that do not exist in a CTP but are commonly used in any message passing program, such as branches, loops, functions, are not supported by the algorithm in this paper. Future work will explore to extend the algorithm in this paper to support these structures. As discussed earlier, the algorithm in this paper does not scale well to a program with wide communication due to how a process is sectioned. Future work will consider new ways to section processes so that programs with various communication types are supported.
