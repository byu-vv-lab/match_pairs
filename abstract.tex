%!TEX root = paper.tex
Message Passing Interface (MPI) is commonly used in high performance computing (HPC). 
There are two send modes that complicates MPI semantics: synchronous send (the send does not return until the matching receive is posted) and ready send (the matching receive must be first posted without an error message). 
A problem in any MPI program, is the violation of send modes.  
A program may be incompatible with synchronous send meaning no schedule exists where each synchronous send returns correctly. A synchronous send incompatible program deadlocks. 
Also, a program execution may throw an error message where a ready send returns before the matching receive is posted.
The problem is difficult due to the message non-determinism where a receive may potentially match with multiple sends. 
%
This paper presents an approach that verifies MPI send modes in two steps: first encoding a concurrent trace program (CTP) into a Satisfiability Modulo Theories (SMT) problem, which if satisfiable, yields a feasible schedule that is sufficient to prove the compatibility of synchronous send; then combining the satisfiable encoding with the intended erroneous behavior of ready sends. If the combined encoding is also satisfiable, a violation of the ready sends is detected.
%
To our knowledge, the new encoding is the first approach that verifies MPI send modes.
% 
As the size of a input match set (each element in the set is a pair of a send and a receive that may potentially match in the runtime) is essential to solving the SMT encoding, this paper presents a new algorithm that under-approximates the match pairs for a CTP by sectioning processes in the CTP. 
%The algorithm has the flexibility to generate the match pair set with various size based on the user input.
The experimental results show that the new encoding with the under-approximation algorithm is able to correctly verify MPI send modes in a set of benchmarks, and is much faster than a existing MPI verifier.



%Message non-determinism makes the error detection in message passing programs very difficult. The prior work uses an over-approximation of the precise match pair records (each is a pair of a send and a receive that may potentially match in the runtime) to capture all possible message communication in a concurrent trace program (CTP). The SMT encoding with such a set of match pairs is able to witness program properties including deadlock, message race, and zero-buffer compatibility, however, is inefficient because of the exponential ways of match pair resolution.
%This paper presents a new algorithm that under-approximates the match pairs for a CTP iteratively: first sectioning each process in the CTP such that each potential sender distributes roughly a bounded number of sends to match the same number of receives in the process, and then approximating the match pairs for the sends and receives in each section independently by a few simple rules with ranking. The algorithm runs in quadratic complexity in the number of operations. Novel in the work is that the algorithm has the flexibility to generate the match pair set with various size based on the user input. This paper further presents that the precise match pairs for any CTP can be generated with a bounded input. The experiments over a set of benchmarks show that the algorithm in this paper drastically reduce the runtime performance of property witnessing as all the properties are witnessed with a small set of match pairs generated by the new algorithm. The results also show that the algorithm is able to scale to a program that employs a high degree of message non-determinism and/or a high degree of deep communication.
